\chapter{}
\section*{Умова:}
Еліптична крива $E$ над полем $\mathbb{F}_{631}$ задана рівнянням:
\begin{equation*}
    y^{2} = x^3 + 30x + 34
\end{equation*}
\begin{enumerate}
    \item Перевірити, що точки $P = (36, 571)$ та $Q = (420, 48)$ належать ЕК;
    \item Перевірити, що обидві точки мають порядок $5$ і породжують підгрупу точок експоненти $5$
    \item Обчислити значення спарювання Вейля $w_{5} (P, Q)$ та перевірити, що отримане значення є коренем 5 
        степеня з 1.
\end{enumerate}

\section*{Розв'язання:}
Перевірку, чи точки справді лежать на кривій ми можемо звичайною підстановкою у рівняння ЕК.

\noindent Для точки $P$:
\begin{align*}
    & P(36, 571) \leftarrow y^{2} = x^3 + 30x + 34 \\
    & 571^{2} = 36^{3} + 30 \cdot 36 + 34 \Mod{631} \\
    & 445 = 445 \Mod{631}
\end{align*}
та точки $Q$:
\begin{align*}
    & Q(420, 48) \leftarrow y^{2} = x^3 + 30x + 34 \\
    & 48^{2} = 420^{3} + 30 \cdot 420 + 34 \Mod{631} \\
    & 441 = 441 \Mod{631}
\end{align*}
Як бачимо, точки належать еліптичній кривій. Для перевірки порядку 5 цих точок, треба обчислити точки $2P$, $3P$, $4P$, 
$5P$ та $2Q$, $3Q$, $4Q$, $5Q$ відповідно і переконатися що $5P = 5Q = \mathcal{O}$. Для зручності обчислень покладемо 
точку на нескінченності рівною $(0, 0)$: $\mathcal{O} = (0, 0)$. Це нам ніщо не забороняє зробити, оскільки $(0, 0)$ не 
належить заданій ЕК.

Суму точок обчислюємо за формулою: %% ! код?
\begin{equation*}
\left\{
    \begin{aligned}
        & x_{P+Q} = \left(\frac{y_1 - y_2}{x_1 - x_2}\right)^{2} - x_1 - x_3 \\
        & y_{P+Q} = \left(\frac{y_1 - y_2}{x_1 - x_2}\right) \cdot \left(x_1 - x_3\right) - y_1
    \end{aligned}
\right.
\end{equation*}
Подвоєння точки:
\begin{equation*}
\left\{
    \begin{aligned}
        & x_{2P} = \left(\frac{3x_{1}^{2} + a}{2y_1}\right)^{2} - 2x_{1} \\
        & y_{2P} = \left(\frac{3x_{1}^{2} + a}{2y_1}\right) \cdot \left(x_1 - x_3\right) - y_1
    \end{aligned}
\right.
\end{equation*}
Мінус точка:
\begin{equation*}
    x_{-P} = x_{P} \quad y_{-P} = - y_{P}
\end{equation*}
Маємо:
\begin{align*}
    & P (36, 571) &\quad & Q (420, 48) \\
    & 2P (617, 5) &\quad & 2Q (121, 387) \\
    & 3P (617, 626) &\quad & 3Q (121, 244) \\
    & 4P (36, 60) &\quad & 4Q (420, 583) \\
    & 5P (0, 0) &\quad & 5Q (0, 0)
\end{align*}

Чи породжують $P$ і $Q$ підгрупу експоненти порядку 5? Для цього візьмемо якусь умовну точку 
$S = \alpha P + \beta Q$ з цієї підгрупи, де $\alpha, \beta \in \mathbb{N} [0, 5)$. Знайдемо всі точки:

\begin{align*}
& P (36, 571), 2P (617, 5), 3P (617, 626), 4P (36, 60) \\
& Q (420, 48), 2Q (121, 387), 3Q (121, 244), 4Q (420, 583) \\
& 5P = 5Q = \mathcal{O} = (0, 0) \text{ -- точка на нескінченності} \\
& P +  Q = (36, 571) + (420, 48) = (575, 7) \\
& P + 2Q = (36, 571) + (121, 387) = (531, 18) \\
& P + 3Q = (36, 571) + (121, 244) = (595, 221) \\
& P + 4Q = (36, 571) + (420, 583) = (289, 269) \\
& 2P +  Q = (617, 5) + (420, 48) = (586, 584) \\
& 2P + 2Q = (617, 5) + (121, 387) = (428, 25) \\
& 2P + 3Q = (617, 5) + (121, 244) = (511, 608) \\
& 2P + 4Q = (617, 5) + (420, 583) = (339, 499) \\
& 3P +  Q = (617, 626) + (420, 48) = (339, 132) \\
& 3P + 2Q = (617, 626) + (121, 387) = (511, 23) \\
& 3P + 3Q = (617, 626) + (121, 244) = (428, 606) \\
& 3P + 4Q = (617, 626) + (420, 583) = (586, 47) \\
& 4P +  Q = (36, 60) + (420, 48) = (289, 362) \\
& 4P + 2Q = (36, 60) + (121, 387) = (595, 410) \\
& 4P + 3Q = (36, 60) + (121, 244) = (531, 613) \\
& 4P + 4Q = (36, 60) + (420, 583) = (575, 624)
\end{align*}
Усього можливих точок підгрупи експоненти 5 -- 25 штук.

\noindent 5, це мається на увазі, що кожен елемент з цієї множини матиме порядок кратний 5. Перевірити це легко, 
можна взяти точку $5S$ і переконатися чи вона рівна $\mathcal{O}$.
\begin{equation*}
    5S =  5 \cdot(\alpha P + \beta Q) = 5 \cdot \alpha P + 5 \cdot \beta Q = \MakeWaveComment{$\substack{\alpha, \beta, 5 \\ \text{звичайні числа}}$} 
    = \alpha \cdot (5P) + \beta \cdot (5Q) = \alpha \cdot \mathcal{O} + \beta \cdot \mathcal{O} = \mathcal{O}.
\end{equation*}

\noindent Бачимо, що кожен елемент у підгрупі зануляється при множенні на 5, тобто ці точки породжують підгрупу 
точок експоненти 5.

Значення спарювання Вейля $w_5(P, Q)$ обчислюємо формулою:
\begin{equation*}
    w_5(P, Q) = \frac{f_{5,P}(Q+R)/ f_{5,P}(R)}{f_{5,Q}(P-R) / f_{5,Q}(-R)}
\end{equation*}
Для цього зафіксуємо точку $R = (0, 36)$, що не лежить в підгрупі (з попереднього списку явно видно, що вона не 
знаходиться), і використавши алгоритм Міллера знайдемо ці чотири невідомі значення:
\begin{equation*}
    f_{5,P}(Q + R), \quad f_{5,P}(R), \quad f_{5,Q}(P - R), \quad f_{5,Q}(-R)
\end{equation*}

\begin{algorithm}[H]
    \caption{Алгоритм Міллера}
    \begin{algorithmic}[1]
        \Require $n \in \mathbb{N}$, $P \in E$, $D$ --- дівізор, для якого $\mathrm{supp}(D) \cap \{P, 0\} = \emptyset$.
        \Ensure $f_{n,P}(D)$.

        \State Знаходимо бінарний розклад числа $n$:
        \begin{equation*}
            n = \sum_{j=0}^{L} n_j \cdot 2^j, \quad n_j \in \{0, 1\}, \quad n_L = 1
        \end{equation*}

        \State \textbf{Ініціалізація:}
        \Statex $T = P$
        \Statex $f = 1$

        \For{($j = L - 1; j \geq 0; j--$)}
        \State $f = f^2 \cdot \frac{I_{T,T}(D)}{v_{[2]T}(D)}$
        \State $T = [2]T$
        \If{$n_j = 1$}
        \State $f = f \cdot \frac{I_{T,P}(D)}{v_{T+P}(D)}$
        \State $T = T + P$
        \EndIf
        \EndFor

        \State \Return $f$
    \end{algorithmic}
\end{algorithm}
\newpage
\begin{itemize}
    \item $I_{P,P}$ -- пряма, яка є дотичною до ЕК в точці $P$.
    \item $I_{P,Q}$ -- пряма, що проходить через точки $P$, $Q$, $P \neq Q$.
    \item $v_{P}$ -- вертикальна пряма через точку $P$
\end{itemize}
Спершу зрозуміємо, як обраховувати кожну з прямих:
\begin{enumerate}
    \item $I_{P,P} (D)$ \\
        Рівняння дотичної до ЕК має вигляд: $y - y_{P} = \lambda \cdot (x - x_{P})$
        Це схоже на формулу для $y$-координати $2P$, але є один нюанс: точка $2P$, на еліптичній кривій -- це 
        відбиття точки перетину дотичної з кривою відносно вісі $X$, тобто $(x_{2P}, -y_{2P}) \to (x_{2P}, y_{2P})$. 
        Якщо підставити точку $2P$ в це рівняння, то можна отримати відому формулу для $y_{2P}$:
        \begin{equation*}
            -y_{2P} - y_{P} = \lambda \cdot (x_{2P} -  x_{P}) \Rightarrow y_{2P} = \lambda \cdot (x_{P} - x_{2P}) - y_{P},
        \end{equation*}
        де $\lambda = \frac{3x_{P}^{2} + a}{2 y_{P}}$.

        Повертаючись до того, з чого починали, замінимо $P$ на $T$, перенесемо все в одну сторону і перегруповуємо:
        $y - y_{T} - \lambda x + \lambda x_{T} = y - \lambda x + \left(\lambda x_{T} - y_{T}\right)$ \\
        Остаточно маємо:
        \begin{equation*}
            I_{T,T} (D) = y - y_{T} - \lambda x + \lambda x_{T} = y_{D} - \lambda x_{D} + \left(\lambda x_{T} - y_{T}\right)
        \end{equation*}
    \item $I_{P,Q} (D)$ \\
        Тут аналогічна ситуація. Рівняння січної до ЕК має вигляд $y - y_{P} = \lambda \cdot (x -  x_{P})$
        Це схоже на формулу для $y$-координати $P+Q$, але такий самий нюанс: точка перетину вже січною еліптичної 
        кривої має відбитися відносно вісі $X$, тобто $(x_{P+Q}, -y_{P+Q}) \to (x_{P+Q}, y_{P+Q})$ -- будуть справжніми 
        координатами $P+Q$. В цьому можна переконатися, якщо підставити точку $P+Q$ в це рівняння, то можна отримати 
        вже формулу для $y_{P+Q}$:
        \begin{equation*}
            -y_{P+Q} - y_{P} = \lambda \cdot (x_{P+Q} -  x_{P}) \Rightarrow y_{P+Q} = \lambda \cdot (x_{P} - x_{P+Q}) - y_{P},
        \end{equation*}
        де $\lambda = \frac{y_{P} - y_{Q}}{x_{P} - y_{Q}}$.

        Повернемось до рівняння січної. Перенесемо все в одну сторону і перегрупуємо:
        $y - y_{T+P} - \lambda x + \lambda x_{T+P} = 0 \Rightarrow y - \lambda x + \left(\lambda x_{T+P} - y_{T+P}\right)$ \\
        Остаточно маємо:
        \begin{equation*}
            I_{T, P} (D) = y_{D} - \lambda x_{D} + \left(\lambda x_{T+P} - y_{T+P}\right)
        \end{equation*}
    \item $v_{P} (D)$ Рівняння вертикальної прямої це: $x - x_{P} = 0$.
    У нас в алгоритмі є два випадки $v_{2T} (D)$ і $v_{T+P} (D)$:
    \begin{itemize}
        \item Якщо $v_{2T} (D)$, то відповідно пряма $x - x_{2T} = 0$, $x_{2T} = 
            \underbrace{\left(\frac{3x_{T}^{2} + a}{2 y_{T}}\right)^{2}}_{\lambda^{2}} - 2 x_{T} = \lambda^{2} - 2 x_{T}$, 
            підставляючи одне в інше, маємо: $v_{2T} (D) = x_{D} - (\lambda^{2} - 2 x_{T})$
        \item Якщо $v_{T+P} (D)$, то пряма має вигляд: $x - x_{T+P} = 0$, 
            $x_{T+P} = \underbrace{\left(\frac{y_{T} - y_{P}}{x_{T} - x_{P}}\right)^{2}}_{\lambda^{2}} - x_{T} - x_{P} = \lambda^{2} - x_{T} - x_{P}$.
            підставляючи одне в інше, маємо: $v_{T+P} (D) = x_{D} - (\lambda^{2} - x_{T} - x_{P})$
    \end{itemize}
\end{enumerate}

\newpage
У нас $n = 5_{10} = 101_{2} \Rightarrow L = 2$ (індекс старшого біту), тобто $L-1 = 1$. Алгоритм матиме $2$ кроки, 
оскільки ініціалізацією $T = P$ ми враховуємо старший біт $n_{L} = 1$.
\begin{enumerate}[label=\textbf{Ітерація \No \theenumi}, leftmargin=*]
    \item
        \begin{align*}
            & I_{T,T} (D) = y_{D} - \lambda x_{D} + \left(\lambda x_{T} - y_{T}\right) \Mod{631}, \quad \lambda = \frac{3x_{T}^{2} + a}{2 y_{T}} \\
            & v_{2T} (D) = x_{D} - (\lambda^{2} - 2 x_{T}) \Mod{631}, \quad \lambda = \frac{3x_{T}^{2} + a}{2 y_{T}} \\
            & f = f^{2} \cdot \frac{I_{T,T} (D)}{V_{[2]T}(D)} \Mod{631} \\
            & T = [2]T, \quad n_{1} = 0, \text{ тому continue} \\
        \end{align*}
    \item
        \begin{align*}
            & I_{T,T} (D) = y_{D} - \lambda x_{D} + \left(\lambda x_{T} - y_{T}\right) \Mod{631}, \quad \lambda = \frac{3x_{T}^{2} + a}{2 y_{T}} \\
            & v_{2T} (D) = x_{D} - (\lambda^{2} - 2 x_{T}) \Mod{631}, \quad \lambda = \frac{3x_{T}^{2} + a}{2 y_{T}} \\
            & f = f^{2} \cdot \frac{I_{T,T} (D)}{V_{[2]T}(D)} \Mod{631} \\
            & T = [2]T, \quad n_{0} = 1, \text{ тому :} \\
            & I_{T, P} (D) = y_{D} - \lambda x_{D} + \left(\lambda x_{T+P} - y_{T+P}\right) \Mod{631}, \quad \lambda = \frac{y_{T} - y_{P}}{x_{T} - y_{P}} \\
            & v_{T+P} (D) = x_{D} - (\lambda^{2} - x_{T} - x_{P}) \Mod{631}, \quad \lambda = \frac{y_{T} - y_{P}}{x_{T} - x_{P}} \\
            & f = f \cdot \frac{I_{T, P} (D)}{v_{T+P} (D)} \Mod{631} \\
            & T = T + P
        \end{align*}
    \item[] \textbf{Return:} $f_{5, P} (D)$
\end{enumerate}

\newpage
\noindent\textbf{\textit{Обчислимо $f_{5,P}(Q + R)$}}

Вхідні дані:

Вхідні дані:
$Q (420, 48)$, $R (0, 36)$, $D = Q + R = (535, 129)$, $a = 30$, $\mathrm{mod} = 631$;

Ініціалізація:
$T = P = (36, 571)$, $f = 1$

\begin{enumerate}[label=\textbf{Ітерація \No\arabic*.}, leftmargin=*]
    \item $j = 1$, $n_1 = 0$
        \begin{align*}
            & \lambda = \frac{3 \cdot 36^2 + 30}{2 \cdot 571} = \frac{3918}{1142} = 132 \cdot 511^{-1} = 62 \Mod{631} \\[1ex]
            & I_{T,T}(Q + R) = 129 - 62 \cdot 535 + (62 \cdot 36 - 571) = 170 \Mod{631} \\[1ex]
            & V_{[2]T}(Q + R) = 535 - (62^2 - 2 \cdot 36) = 549 \Mod{631} \\[1ex]
            & f = 1^2 \cdot \frac{170}{549} = 170 \cdot 549^{-1} = 198 \Mod{631} \\[1ex]
            & [2]T = (617, 5) \\[1ex]
            & T = (617, 5), \quad n_1 = 0, \text{ тому continue}
        \end{align*}
    \item $j = 0$, $n_0 = 1$
        \begin{align*}
            & \lambda = \frac{3 \cdot 617^2 + 30}{2 \cdot 5} = \frac{1142097}{10} = 618 \cdot 10^{-1} = 188 \Mod{631} \\[1ex]
            & I_{T,T}(Q + R) = 129 - 188 \cdot 535 + (188 \cdot 617 - 5) = 396 \Mod{631} \\[1ex]
            & V_{[2]T}(Q + R) = 535 - (188^2 - 2 \cdot 617) = 499 \Mod{631} \\[1ex]
            & f = 198^2 \cdot \frac{396}{499} = 82 \cdot 396 \cdot 499^{-1} = 291 \cdot 435 = 385 \Mod{631} \\[1ex]
            & [2]T = (36, 60) = -P \\[1ex]
            & T = (36, 60), \quad n_0 = 1, \text{ тому:} \\[1ex]
            & \textbf{Особливий випадок: } T + P = \mathcal{O} \text{ (точка на нескінченності)} \\[1ex]
            & I_{T,P}(Q + R) = x_{Q + R} - x_{T} = 535 - 36 = 499 \Mod{631} \quad \text{(вертикальна лінія)} \\[1ex]
            & V_{T+P}(Q + R) = 1 \quad \text{ (для точки $\mathcal{O}$ покладають таке)} \\[1ex]
            & f = 385 \cdot \frac{499}{1} = 385 \cdot 499 = 291 \Mod{631}
        \end{align*}
\end{enumerate}

\textbf{Результат:} $f_{5,P}(Q + R) = 291$

\newpage
\noindent\textbf{\textit{Обчислимо $f_{5,P}(R)$}}

Вхідні дані:
$D = R = (0, 36)$, $a = 30$, $\mathrm{mod} = 631$;

Ініціалізація:
$T = P = (36, 571)$, $f = 1$

\begin{enumerate}[label=\textbf{Ітерація \No\arabic*.}, leftmargin=*]
    \item $j = 1$, $n_1 = 0$
        \begin{align*}
            & \lambda = \frac{3 \cdot 36^2 + 30}{2 \cdot 571} = \frac{3918}{1142} = 132 \cdot 511^{-1} = 62 \Mod{631} \\[1ex]
            & I_{T,T}(R) = 36 - 62 \cdot 0 + (62 \cdot 36 - 571) = 435 \Mod{631} \\[1ex]
            & V_{[2]T}(R) = 0 - (62^2 - 72) = 14 \Mod{631} \\[1ex]
            & f = 1^2 \cdot \frac{435}{14} = 435 \cdot 14^{-1} = 617 \Mod{631} \\[1ex]
            & [2]T = (617, 5) \\[1ex]
            & T = (617, 5), \quad n_1 = 0, \text{ тому continue}
        \end{align*}
    \item $j = 0$, $n_0 = 1$
        \begin{align*}
            & \lambda = \frac{3 \cdot 617^2 + 30}{2 \cdot 5} = \frac{1142097}{10} = 618 \cdot 10^{-1} = 188 \Mod{631} \\[1ex]
            & I_{T,T}(R) = 36 - 188 \cdot 0 + (188 \cdot 617 - 5) = 554 \Mod{631} \\[1ex]
            & V_{[2]T}(R) = 0 - (188^2 - 2 \cdot 617) = 595 \Mod{631} \\[1ex]
            & f = 617^2 \cdot \frac{554}{595} = 196 \cdot 554 \cdot 595^{-1} = 52 \cdot 333 = 279 \Mod{631} \\[1ex]
            & [2]T = (36, 60) = -P \\[1ex]
            & T = (36, 60), \quad n_0 = 1, \text{ тому:} \\[1ex]
            & \textbf{Особливий випадок: } T + P = \mathcal{O} \\[1ex]
            & I_{T,P}(R) = x_{R} - x_T = 0 - 36 = 595 \Mod{631} \\[1ex]
            & V_{T+P}(R) = 1 \\[1ex]
            & f = 279 \cdot \frac{595}{1} = 52 \Mod{631}
        \end{align*}
\end{enumerate}

\textbf{Результат:} $f_{5,P}(R) = 52$

\newpage
\noindent\textbf{\textit{Обчислимо $f_{5,Q}(P - R)$}}

Вхідні дані:
$P (36, 571)$, $R (0, 36)$, $D = P - R = P + (-R) = (315, 246)$, $a = 30$, $\mathrm{mod} = 631$;

Ініціалізація:
$T = Q = (420, 48)$, $f = 1$

\begin{enumerate}[label=\textbf{Ітерація \No\arabic*.}, leftmargin=*]
    \item $j = 1$, $n_1 = 0$
        \begin{align*}
            & \lambda = \frac{3 \cdot 420^2 + 30}{2 \cdot 48} = \frac{529230}{96} = 452 \cdot 96^{-1} = 31 \Mod{631} \\[1ex]
            & I_{T,T}(P - R) = 246 - 31 \cdot 315 + (31 \cdot 420 - 48) = 298 \Mod{631} \\[1ex]
            & V_{[2]T}(P - R) = 315 - (31^2 - 840) = 194 \Mod{631} \\[1ex]
            & f = 1^2 \cdot \frac{298}{194} = 298 \cdot 194^{-1} = 587 \Mod{631} \\[1ex]
            & [2]T = (121, 387) \\[1ex]
            & T = (121, 387), \quad n_1 = 0, \text{ тому continue}
        \end{align*}
    \item $j = 0$, $n_0 = 1$
        \begin{align*}
            & \lambda = \frac{3 \cdot 121^2 + 30}{2 \cdot 387} = \frac{43953}{774} = 414 \cdot 143^{-1} = 250 \Mod{631} \\[1ex]
            & I_{T,T}(P - R) = 246 - 250 \cdot 315 + (250 \cdot 121 - 387) = 577 \Mod{631} \\[1ex]
            & V_{[2]T}(P - R) = 315 - (250^2 - 242) = 526 \Mod{631} \\[1ex]
            & f = 587^2 \cdot \frac{577}{526} = 43 \cdot 577 \cdot 526^{-1} = 202 \cdot 6 = 581 \Mod{631} \\[1ex]
            & [2]T = (420, 583) = -Q \\[1ex]
            & T = (420, 583), \quad n_0 = 1, \text{ тому:} \\[1ex]
            & \textbf{Особливий випадок: } T + Q = \mathcal{O} \\[1ex]
            & I_{T,Q}(P - R) = x_{P - R} - x_T = 315 - 420 = 526 \Mod{631} \\[1ex]
            & V_{T+Q}(P - R) = 1 \\[1ex]
            & f = 581 \cdot \frac{526}{1} = 202 \Mod{631}
        \end{align*}
\end{enumerate}

\textbf{Результат:} $f_{5,Q}(P - R) = 202$

\newpage
\noindent\textbf{\textit{Обчислимо $f_{5,Q}(-R)$}}

Вхідні дані:
$D = -R = (0, 595)$, $a = 30$, $\mathrm{mod} = 631$;

Ініціалізація:
$T = Q = (420, 48)$, $f = 1$

\begin{enumerate}[label=\textbf{Ітерація \No\arabic*.}, leftmargin=*]
    \item $j = 1$, $n_1 = 0$
        \begin{align*}
            & \lambda = \frac{3 \cdot 420^2 + 30}{2 \cdot 48} = \frac{529230}{96} = 452 \cdot 96^{-1} = 31 \Mod{631} \\[1ex]
            & I_{T,T}(-R) = 595 - 31 \cdot 0 + (31 \cdot 420 - 48) = 316 \Mod{631} \\[1ex]
            & V_{[2]T}(-R) = 0 - (31^{2} - 840) = 510 \Mod{631} \\[1ex]
            & f = 1^2 \cdot \frac{316}{510} = 316 \cdot 510^{-1} = 352 \Mod{631} \\[1ex]
            & [2]T = (121, 387) \\[1ex]
            & T = (121, 387), \quad n_1 = 0, \text{ тому continue}
        \end{align*}
    \item $j = 0$, $n_0 = 1$
        \begin{align*}
            & \lambda = \frac{3 \cdot 121^2 + 30}{2 \cdot 387} = \frac{43953}{774} = 414 \cdot 143^{-1} = 250 \Mod{631} \\[1ex]
            & I_{T,T}(-R) = 595 - 250 \cdot 0 + (250 \cdot 121 - 387) = 170 \Mod{631} \\[1ex]
            & V_{[2]T}(-R) = 0 - (250^{2} - 774) = 211 \Mod{631} \\[1ex]
            & f = 352^2 \cdot \frac{170}{211} = 228 \cdot 170 \cdot 211^{-1} = 269 \cdot 317 = 88 \Mod{631} \\[1ex]
            & [2]T = (420, 583) = -Q \\[1ex]
            & T = (420, 583), \quad n_0 = 1, \text{ тому:} \\[1ex]
            & \textbf{Особливий випадок: } T + Q = \mathcal{O} \\[1ex]
            & I_{T,Q}(-R) = x_D - x_T = 0 - 420 = 211 \Mod{631} \\[1ex]
            & V_{T+Q}(-R) = 1 \\[1ex]
            & f = 88 \cdot \frac{211}{1} = 269 \Mod{631}
        \end{align*}
\end{enumerate}

\textbf{Результат:} $f_{5,Q}(-R) = 269$

\noindent\textbf{\textit{Обчислюємо спарювання Вейля}}

Спираючись на попередні обчислення:
\begin{equation*}
    w_5(P, Q) = \frac{f_{5,P}(Q+R)/f_{5,P}(R)}{f_{5,Q}(P-R) / f_{5,Q}(-R)} = \frac{291/52}{202/269} = 
    \frac{291 \cdot 449}{202 \cdot 441} = \frac{42}{111} = 42 \cdot 523 = 512 \Mod{631}
\end{equation*}

\textbf{Перевірка:} $512^{5} \Mod{631} = 1 \Rightarrow$ є коренем 5 степеня з 1.
