\chapter{}
\section*{Умова:}
Еліптична крива $E$ над полем $\mathbb{F}_{631}$ задана рівнянням:
\begin{equation*}
    y^{2} = x^3 + 30x + 34
\end{equation*}
\begin{enumerate}
    \item Перевірити, що точки $P = (36, 571)$ та $Q = (420, 48)$ належать ЕК;
    \item Перевірити, що обидві точки мають порядок $5$ і породжують підгрупу точок експоненти $5$
    \item Обчислити значення спарювання Вейля $w_{5} (P, Q)$ та перевірити, що отримане значення є коренем 5 степеня з 1.
\end{enumerate}

\section*{Розв'язання:}
Перевірку, чи точки справді лежать на кривій ми можемо звичайною підстановкою у рівняння ЕК.

\noindent Для точки $P$:
\begin{align*}
    & P(36, 571) \leftarrow y^{2} = x^3 + 30x + 34 \\
    & 571^{2} = 36^{3} + 30 \cdot 36 + 34 \Mod{631} \\
    & 445 = 445 \Mod{631}
\end{align*}
та точки $Q$:
\begin{align*}
    & Q(420, 48) \leftarrow y^{2} = x^3 + 30x + 34 \\
    & 48^{2} = 420^{3} + 30 \cdot 420 + 34 \Mod{631} \\
    & 441 = 441 \Mod{631}
\end{align*}
Як бачимо, точки належать еліптичній кривій. Для перевірки порядку 5 цих точок, треба обчислити точки $2P$, $3P$, $4P$, 
$5P$ та $2Q$, $3Q$, $4Q$, $5Q$ відповідно і переконатися що $5P = 5Q = \mathcal{O}$. Для зручності обчислень покладемо 
точку на нескінченності рівною $(0, 0)$: $\mathcal{O} = (0, 0)$. Це нам ніщо не забороняє зробити, оскільки $(0, 0)$ не належить заданій 
ЕК.
