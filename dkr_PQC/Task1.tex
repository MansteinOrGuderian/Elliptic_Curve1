\chapter{}
\section*{Умова:}
\begin{enumerate}
    \item Перевірити, що набір публічних параметрів $(N, p, q, d) = (7, 3, 41, 2)$ криптосистеми NTRUCrypt є коректним;
    \item Використовуючи публічний ключ:
        \begin{equation*}
            pk = 18 x^{6} + 6 x^{5} - 11 x^{4} + 3 x^{3} - 15 x^{2} - 2x - 2 = h(x)
        \end{equation*}
        зашифрувати повідомлення $m = -x^{5} - x^{4} - x^{2} - x + 1$.
\end{enumerate}
Спершу перевіримо, що набір параметрів є коректним. Це робиться дуже просто. Треба усього лиш перевірити, що виконується 
наступна рівність: $q > (6d + 1) p$.

\noindent У нас задано:
\begin{itemize}
    \item $N = 7$;
    \item $p = 3$;
    \item $q = 41$;
    \item $d = 2$.
\end{itemize}
Маємо: $41 > (6 \cdot 2 + 1) \cdot 3 = 39$. Справді, $41 > 39$, то ж система NTRUCrypt є коректною.

Далі займемося зашифруванням повідомлення. Для цього нам необхідно спершу обчислити центральне підняття $m$. В нашому 
випадку саме повідомлення $m = -x^{5} - x^{4} - x^{2} - x + 1$ і буде центральним підняттям, оскільки коефіцієнти 
вже лежать у напівінтервалі $(-41/2, 41/2]$, тобто в проміжку $(-20, 20]$.

Наступним нашим кроком є обрання випадкового многочлена $\mathbf{r}$ з множини тернарних многочленів $\mathcal{T} (d, d)$. 
В нас це $\mathcal{T} (2, 2)$ -- де перша двійка вказує на кількість коефіцієнтів $1$, а друга відповідно коефіцієнтів 
$-1$.

Нехай це буде многочлен $\mathbf{r} = x^{6} - x^{5} + x - 1$. Щоб отримати шифротекст $\mathbf{e}(x)$ необхідно обчислити:
\begin{equation*}
    \mathbf{e}(x) = p \cdot \mathbf{r}(x) \star \mathbf{h}(x) + \mathbf{m}(x)
\end{equation*}
Зробимо покроково. Спершу обчислимо добуток a.k.a. згортковий многочлен:
\begin{equation*}
    \mathbf{r}(x) \star \mathbf{h}(x) = \left(x^{6} - x^{5} + x - 1\right) \star \left(18 x^{6} + 6 x^{5} - 11 x^{4} 
    + 3 x^{3} - 15 x^{2} - 2x - 2\right)
\end{equation*}
Для цього необхідно обчислити результуючі коефіцієнти. Скористаємось формулою:
\begin{equation*}
    c_{k} = \sum\limits_{i + j \equiv k \Mod{N}} a_{i} \cdot b_{j}, \quad 0 \leq k \leq N-1
\end{equation*}
Для нашої задачі $N = 6$, тому формула матиме вигляд:
\begin{equation*}
    c_{k} = \sum\limits_{i + j \equiv k \Mod{7}} a_{i} \cdot b_{j}, \quad 0 \leq k \leq 6
\end{equation*}
Обчислимо коефіцієнти $c_{0}, \dots, c_{6}$, маючи
\begin{align*}
    & a_0 = -1, \, a_1 = 1, \, a_2 = 0, \, a_3 = 0, \, a_4 = 0, \, a_5 = -1, \, a_6 = 1 \\
    & b_0 = -2, \, b_1 = -2, \,  b_2 = -15, \, b_3 = 3, \, b_4 = -11, \, b_5 = 6, \, b_6 = 18
\end{align*}
\begin{align*}
    & c_{0} = \sum\limits_{i + j \equiv 0 \Mod{7}} a_{i} \cdot b_{j} = a_{0} \cdot b_{0} + a_{1} \cdot b_{6} + a_{2} \cdot b_{5} + a_{3} \cdot b_{4} + a_{4} \cdot b_{3} + a_{5} \cdot b_{2} + a_{6} \cdot b_{1} = \\
    & = a_{0} \cdot b_{0} + a_{1} \cdot b_{6} + a_{5} \cdot b_{2} + a_{6} \cdot b_{1} = (-1) \cdot (-2) + 1 \cdot 18 + (-1) \cdot (-15) + 1 \cdot (-2) = \\
    & = 33 \Mod{41} \\
    & c_{1} = \sum\limits_{i + j \equiv 1 \Mod{7}} a_{i} \cdot b_{j} = a_{0} \cdot b_{1} + a_{1} \cdot b_{0} + a_{2} \cdot b_{6} + a_{3} \cdot b_{5} + a_{4} \cdot b_{4} + a_{5} \cdot b_{3} + a_{6} \cdot b_{2} = \\
    & = a_{0} \cdot b_{1} + a_{1} \cdot b_{0} + a_{5} \cdot b_{3} + a_{6} \cdot b_{2} = (-1) \cdot (-2) + 1 \cdot (-2) + (-1) \cdot 3 + 1 \cdot (-15) = -18 = \\
    & = 23 \Mod{41} \\
    & c_{2} = \sum\limits_{i + j \equiv 2 \Mod{7}} a_{i} \cdot b_{j} = a_{0} \cdot b_{2} + a_{1} \cdot b_{1} + a_{2} \cdot b_{0} + a_{3} \cdot b_{6} + a_{4} \cdot b_{5} + a_{5} \cdot b_{4} + a_{6} \cdot b_{3} = \\
    & = a_{0} \cdot b_{2} + a_{1} \cdot b_{1} + a_{5} \cdot b_{4} + a_{6} \cdot b_{3} = (-1) \cdot (-15) + 1 \cdot (-2) + (-1) \cdot (-11) + 1 \cdot 3 = \\
    & = 27 \Mod{41} \\
    & c_{3} = \sum\limits_{i + j \equiv 3 \Mod{7}} a_{i} \cdot b_{j} = a_{0} \cdot b_{3} + a_{1} \cdot b_{2} + a_{2} \cdot b_{1} + a_{3} \cdot b_{0} + a_{4} \cdot b_{6} + a_{5} \cdot b_{5} + a_{6} \cdot b_{4} = \\
    & = a_{0} \cdot b_{3} + a_{1} \cdot b_{2} + a_{5} \cdot b_{5} + a_{6} \cdot b_{4} = (-1) \cdot 3 + 1 \cdot (-15) + (-1) \cdot 6 + 1 \cdot (-11) = -35 = \\
    & = 6 \Mod{41} \\
    & c_{4} = \sum\limits_{i + j \equiv 4 \Mod{7}} a_{i} \cdot b_{j} = a_{0} \cdot b_{4} + a_{1} \cdot b_{3} + a_{2} \cdot b_{2} + a_{3} \cdot b_{1} + a_{4} \cdot b_{0} + a_{5} \cdot b_{6} + a_{6} \cdot b_{5} = \\
    & = a_{0} \cdot b_{4} + a_{1} \cdot b_{3} + a_{5} \cdot b_{6} + a_{6} \cdot b_{5} = (-1) \cdot (-11) + 1 \cdot 3 + (-1) \cdot 18 + 1 \cdot 6 = \\
    & = 2 \Mod{41} \\
    & c_{5} = \sum\limits_{i + j \equiv 5 \Mod{7}} a_{i} \cdot b_{j} = a_{0} \cdot b_{5} + a_{1} \cdot b_{4} + a_{2} \cdot b_{3} + a_{3} \cdot b_{2} + a_{4} \cdot b_{1} + a_{5} \cdot b_{0} + a_{6} \cdot b_{6} = \\
    & = a_{0} \cdot b_{5} + a_{1} \cdot b_{4} + a_{5} \cdot b_{0} + a_{6} \cdot b_{6} = (-1) \cdot 6 + 1 \cdot (-11) + (-1) \cdot (-2) + 1 \cdot 18 = \\
    & = 3 \Mod{41} \\
    & c_{6} = \sum\limits_{i + j \equiv 6 \Mod{7}} a_{i} \cdot b_{j} = a_{0} \cdot b_{6} + a_{1} \cdot b_{5} + a_{2} \cdot b_{4} + a_{3} \cdot b_{3} + a_{4} \cdot b_{2} + a_{5} \cdot b_{1} + a_{6} \cdot b_{0} = \\
    & = a_{0} \cdot b_{6} + a_{1} \cdot b_{5} + a_{5} \cdot b_{1} + a_{6} \cdot b_{0} = (-1) \cdot 18 + 1 \cdot 6 + (-1) \cdot (-2) + 1 \cdot (-2) = -12 = \\
    & = 29 \Mod{41}
\end{align*}
Отже маємо:
\begin{equation*}
    c(x) = \mathbf{r}(x) \star \mathbf{h}(x) = 29 x^{6} + 3 x^{5} + 2 x^{4} + 6 x^{3} + 27 x^{2} + 23 x + 33
\end{equation*}
Далі помножимо на $p$:
\begin{align*}
    & p \cdot c(x) = 3 \cdot \left(29 x^{6} + 3 x^{5} + 2 x^{4} + 6 x^{3} + 27 x^{2} + 23 x + 33\right) = \\
    & = 5 x^{6} + 9 x^{5} + 6 x^{4} + 18 x^{3} + 40 x^{2} + 28 x + 17 \Mod{41}
\end{align*}
Далі додамо $\mathbf{m}(x)$ і отримаємо зашифроване повідомлення:
\begin{align*}
    & p \cdot c(x) + \mathbf{m}(x) = 5 x^{6} + 9 x^{5} + 6 x^{4} + 18 x^{3} + 40 x^{2} + 28 x + 17 -x^{5} - x^{4} - x^{2} - x + 1 = \\
    & = 5 x^{6} + 8 x^{5} + 5 x^{4} + 18 x^{3} + 39 x^{2} + 27 x + 18 \Mod{41}
\end{align*}
Отриманий криптотекст має вигляд:
\begin{equation*}
    \mathbf{e}(x) = p \cdot \mathbf{r}(x) \star \mathbf{h}(x) + \mathbf{m}(x) = 5 x^{6} + 8 x^{5} + 5 x^{4} + 18 x^{3} + 39 x^{2} + 27 x + 18
\end{equation*}
